\subsection*{\href{https://travis-ci.com/ajeetwankhede/Discrete-planner-for-motion-planning}\texttt{ } }

\subsection*{Project Overview}

2D discrete path planners are designed for the navigation of a holonomic robot in a known static and flat environment. All the planners share the same common interface and should implement the ​search​ method that has the following signature\+: search(world\+\_\+state, robot\+\_\+pose, goal\+\_\+pose) return path Inputs\+: world\+\_\+state is a 2D-\/grid representation of the environment where the value 0 indicates a navigable space and the value 1 indicates an occupied/obstacle space. robot\+\_\+pose is a tuple of two indices (x, y) which represent the current pose of the robot in world\+\_\+state​. goal\+\_\+pose​ ​is a tuple of two indices (x, y) which represent the goal in ​world\+\_\+state coordinate system. \mbox{\hyperlink{classOutput}{Output}}\+: path is a list of tuple (x, y) representing a path from the ​robot\+\_\+pose to the goal\+\_\+pose​ in world\+\_\+state​ or​ None​ if no path has been found. Two discrete planners are implemented\+:


\begin{DoxyEnumerate}
\item Random Planner The random planner tries to find a path to the goal by randomly moving in the environment (only orthogonal moves are legal). If the planner can not find an acceptable solution in less than max\+\_\+step\+\_\+number, the search should fail. The random planner, while being erratic, has a short memory, and it will never attempt to visit a cell that was visited in the last ​sqrt(max\+\_\+step\+\_\+number)​ steps except if this is the only available option.
\item Optimal Planner A 2D path planner with A$\ast$ algorithm is designed and developed, for the navigation of the robot (only orthogonal moves are legal). The A$\ast$ algorithm ensures optimality of the path, with obstacle avoidance defined in the world.
\end{DoxyEnumerate}

The performance of these two planners is compared on basis of search space explored, time taken to find a path, and optimality of the path found. Random planner was run 10 times and optimal planner once, and the results are summarized in the following table.

\subsection*{Link for S\+IP document}

\mbox{[}S\+IP Document\mbox{]}\href{https://docs.google.com/spreadsheets/d/1OKOs_5UbBNU4q0WjCL7nKXChtRCFgHgl8hptJVIeyFo/edit?usp=sharing}\texttt{ https\+://docs.\+google.\+com/spreadsheets/d/1\+O\+K\+Os\+\_\+5\+Ub\+B\+N\+U4q0\+Wj\+C\+L7n\+K\+X\+Cht\+R\+C\+Fg\+Hgl8hpt\+J\+V\+Iey\+Fo/edit?usp=sharing}

\subsection*{Dependencies}

The path planning module has following dependencies\+:
\begin{DoxyEnumerate}
\item googletest
\item cmake
\item gnuplot
\item \href{http://stahlke.org/dan/gnuplot-iostream/}\texttt{ gnuplot-\/iostream}
\end{DoxyEnumerate}

\#\# Standard install via command-\/line 
\begin{DoxyCode}{0}
\DoxyCodeLine{git clone --recursive https://github.com/ajeetwankhede/Discrete-planner-for-motion-planning}
\DoxyCodeLine{cd <path to repository>}
\DoxyCodeLine{mkdir build}
\DoxyCodeLine{cd build}
\DoxyCodeLine{cmake ..}
\DoxyCodeLine{make}
\DoxyCodeLine{Run tests: ./test/cpp-test}
\DoxyCodeLine{Run program: ./app/shell-app}
\end{DoxyCode}


\subsection*{Run a demo}

After running the steps for standard install via comman-\/line, run the program. The program asks whether the user wants to run a demo. Type \textquotesingle{}m\textquotesingle{} for a demo and then select any planner. The demo example should generate following figures.


\begin{DoxyEnumerate}
\item Random planner demo\+:
\end{DoxyEnumerate}

 


\begin{DoxyEnumerate}
\item Optimal planner demo\+:
\end{DoxyEnumerate}

 

\subsection*{Building for code coverage}

The code coverage of the functions is 82.\+1\% and to get a .html output run the following commands. 
\begin{DoxyCode}{0}
\DoxyCodeLine{sudo apt-get install lcov}
\DoxyCodeLine{cd <path to repository>}
\DoxyCodeLine{mkdir build}
\DoxyCodeLine{cd build}
\DoxyCodeLine{cmake -D COVERAGE=ON -D CMAKE\_BUILD\_TYPE=Debug ../}
\DoxyCodeLine{make}
\DoxyCodeLine{make code\_coverage}
\end{DoxyCode}
 This generates a index.\+html page in the build/coverage sub-\/directory that can be viewed locally in a web browser. Screenshot of the .html file is shown below.

 

\subsection*{How to generate Doxygen report}


\begin{DoxyCode}{0}
\DoxyCodeLine{sudo apt-get install doxygen}
\DoxyCodeLine{cd <path to repository>}
\DoxyCodeLine{mkdir Doxygen}
\DoxyCodeLine{cd Doxygen}
\DoxyCodeLine{doxygen -g <config\_file\_name>}
\DoxyCodeLine{gedit <config\_file\_name>}
\end{DoxyCode}
 Update P\+R\+O\+J\+E\+C\+T\+\_\+\+N\+A\+ME and I\+N\+P\+UT fields in the configuration file. Then run the following command to generate the documentations. 
\begin{DoxyCode}{0}
\DoxyCodeLine{doxygen <config\_file\_name>}
\end{DoxyCode}
 In Doxygen folder, config file and genertaed reports are saved. 